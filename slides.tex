\documentclass{beamer}
\usepackage[utf8x]{inputenc}
\usepackage[T1]{fontenc}
\usepackage{listings}
\usepackage{longtable}
\usepackage{hyperref}

\hypersetup{colorlinks=true,urlcolor=blue}

\usetheme{Warsaw}
\title[Introduction to R]{A primer in R for Stata and SPSS users}
\author{Cole Beck}
\date{April 8, 2014}
\begin{document}

\begin{frame}
\titlepage
\end{frame}

\begin{frame}{Why R?}

Pros of Stata, SPSS, SAS

\begin{enumerate}
\item User friendly
\item Easy to experiment
\item Big data
\end{enumerate}

Pros of R

\begin{enumerate}
\item Access to recently developed research methods

\begin{itemize}
\item nbpMatching package only available with R
\end{itemize}
\item Reproducibility
\item Customizable plots
\item Availability of source code
\item Free outside academia
\end{enumerate}

Biggest Con of R

\begin{itemize}
\item Learning curve of command-line tools

\begin{itemize}
\item RStudio helps
\end{itemize}
\end{itemize}

\end{frame}

\begin{frame}{Examining RStudio}

Setting the working directory, the default location for reading and writing files

\begin{itemize}
\item Session -\textgreater{} Set Working Directory -\textgreater{} Choose Directory\ldots{}
\end{itemize}

See: ?setwd, ?getwd

\begin{itemize}
\item Panels (RStudio -\textgreater{} Preferences\ldots{} -\textgreater{} Pane Layout)
\end{itemize}

\begin{enumerate}
\item Source: R script containing code
\item Console: Interactive R session
\item Environment: List of variables and functions
\item History: History of commands from interactive session
\item Files: Built-in file manager
\item Plots: Generated plots
\item Packages: Package management
\item Help: Documentation
\item Viewer: View local web content
\end{enumerate}

\end{frame}

\begin{frame}{Importing Datasets}

\begin{itemize}
\item Delimited, fixed-width, binary files

\begin{itemize}
\item CSV (Comma separated values) are common
\item Tab-delimited, and others
\item Fixed-width need column widths
\item Options for Stata and SPSS
\end{itemize}
\item Environment pane: Import Dataset

\begin{itemize}
\item Import from file or URL
\end{itemize}
\item Examine column information
\item View dataset
\end{itemize}

\href{http://cran.r-project.org/doc/manuals/r-release/R-data.html}{R Data Import/Export}

\end{frame}

\begin{frame}{Excerpt regarding SPSS and Stata}

\begin{quotation}
Function read.spss can read files created by the ‘save’ and ‘export’ commands in SPSS.
It returns a list with one component for each variable in the saved data set.
SPSS variables with value labels are optionally converted to R factors.

Stata .dta files are a binary file format.
Files from versions 5 up to 11 of Stata can be read and written by functions read.dta and write.dta.
Stata variables with value labels are optionally converted to (and from) R factors.
Stata version 12 by default writes ‘format-115 datasets’: read.dta currently may not be able to read those.
\end{quotation}

\end{frame}

\begin{frame}{Variables}

\begin{itemize}
\item Numbers: numeric
\item Strings: character
\item Boolean: logical (True and False)
\item Datasets: data.frame, matrix
\item Vectors and Lists
\end{itemize}

\end{frame}

\begin{frame}{Functions}

\begin{itemize}
\item Built-in commands
\item User-defined
\item Arguments
\item Tab-completion
\end{itemize}

See: ?apropos

\end{frame}

\begin{frame}{Packages}

\begin{itemize}
\item Packages pane: load and install
\end{itemize}

See: ?library, ?require, ?install.packages, ?update.packages, ?remove.packages

\begin{itemize}
\item Help pane: help is available
\end{itemize}

Example: help(package=''foreign'')

\end{frame}

\begin{frame}{Plots}

\begin{itemize}
\item Plots pane: export with specified format and size
\end{itemize}

\end{frame}

\end{document}
